\documentclass[11pt]{article}
\usepackage{geometry}\geometry{margin=1in}
\usepackage{amsmath, amssymb, amscd, amsthm, amsfonts}
\usepackage{todonotes, xspace}
\usepackage{enumitem}
\usepackage{tabularx}
\usepackage{multirow}

\usepackage{algorithm,algpseudocode}
% \usepackage{graphicx}
% \usepackage{hyperref}

\usepackage{xcolor}
\usepackage{listings}

\definecolor{mGreen}{rgb}{0,0.6,0}
\definecolor{mGray}{rgb}{0.5,0.5,0.5}
\definecolor{mPurple}{rgb}{0.58,0,0.82}
\definecolor{backgroundColour}{rgb}{0.95,0.95,0.92}

\lstdefinestyle{CStyle}{
    backgroundcolor=\color{backgroundColour},
    commentstyle=\color{mGreen},
    keywordstyle=\color{magenta},
    numberstyle=\tiny\color{mGray},
    stringstyle=\color{mPurple},
    basicstyle=\footnotesize,
    breakatwhitespace=false,
    breaklines=true,
    captionpos=b,
    keepspaces=true,
    numbers=left,
    numbersep=5pt,
    showspaces=false,
    showstringspaces=false,
    showtabs=false,
    tabsize=4,
    language=C
}


\newcommand{\EH}[2][inline]{\todo[color=orange!30,#1]{\sf \small\textbf{TODO:} #2\normalsize}\xspace}

\newcommand{\auton}{\textsc{FindTracklets}\xspace}

% \oddsidemargin 0pt
% \evensidemargin 0pt
% \marginparwidth 40pt
% \marginparsep 10pt
% \topmargin -20pt
% \headsep 10pt
% \textheight 8.7in
% \textwidth 6.65in
\linespread{1}

\title{Mini-Project: Finding the Minimum Cost Path in an $N \times N$ Grid}

\author{Evan Hataishi\\ \small ICS 674: Evolutionary Computation\\ \small Prof. L. Altenberg}

\date{\small February 18, 2020}

\begin{document}

\maketitle

% \begin{abstract}
%
%     Given time series detections of moving objects in space, astronomers would like to identify asteroids and link those detections into a linear path within and between nights. Current algorithms use a series of 3-dimensional orthogonal range queries to link neighboring observations into such paths. A fast, efficient, and scalable data structure is needed to reduce the cost of these queries. In this work, we describe the currently used software which uses kd-trees. We then implement the theoretically faster range tree in an attempt to improve the overall performance. Using the same asteroid detection and linking software, we benchmark the kd-tree against our range tree implementation. We find that although just as accurate, the range tree performs much worse in practice for both tree construction time and query time compared to the kd-tree.
%
% \end{abstract}


\section{Introduction}
\label{sec-introduction}

Finding the optimal Binary Search Tree (BST) is a somewhat classical programming problem. Given a set of keys and a probability that each key will be looked up i.e. frequency, the optimal BST is structured s.t. the expected search time over all keys is minimized. The expected cost of searching for any arbitrary key is $frequency \times (depth + 1)$. The expected search time of the entire BST is simply the sum of the search costs of all keys. Specifically, the goal is to construct a BST that minimizes the expected search time, while maintaining BST properties given the key values. Thus, it is preferable to have keys with higher frequencies at a shorter depth. Figure~\ref{fig:opt-bst} shows 5 different BSTs that can be constructed given a set of keys (value row).

\begin{figure}[h]
    \centering
    \includegraphics[width=0.9\columnwidth]{figures/opt_bst.png}
    \caption{ \small Various BST structures constructed from the same keys (reproduced from~\cite{bst_figure}).}
    \label{fig:opt-bst}
\end{figure}

A solution to the optimal BST problem is to simply try each key as root and recursively build each subtree. After exhaustively constructing every valid BST structure, the one with the minimum cost can be chosen. Given a tree with $n$ nodes, this would take exponential time to solve. However, this naive solution can be further improved to $O(n^3)$ with dynamic programming. Although finding the optimal BST structure for any arbitrary set of keys and frequencies has been bounded, we would like to explore the application of an evolutionary algorithm to this problem.

\subsection{Problem Statement}

Given a set of keys $k_1,k_2,\dots,k_n$ and a set of frequencies $f_1,f_2,\dots,f_n$ s.t. $f_i$ is the expected search frequency of $k_i$, we would like to use an evolutionary algorithm to construct a BST with minimal expected search time over all keys given the frequencies.



\section{Evolutionary Algorithm}
\label{sec-algorithm}

\begin{description}
    \item[Search Space] All valid BST structures given a set of keys.
    \item[Representation] An $N$ node BST constructed from a set of $N$ keys and a corresponding set of $N$ frequencies. All BSTs in the search space have the same keys with corresponding frequencies.
    \item[Objective Function] Let $k$ be the set of $N$ keys and $f$ be the set of $N$ frequencies. The expected cost (and, consequently, fitness function) of the BST is calculated as follows:
    \begin{equation*}
        f(k, f) = \sum_{i=1}^{N} (depth(k_i) + 1) \times f_i.
    \end{equation*}
    Therefore, we aim to minimize the expected cost, which will be some non-zero value depending on keys and frequencies.
    \item[Selection Operator] We use elitism to pick the top 30\% of a generation of trees to move on automatically to the next generation. The other 70\% is formed from recombination.
    \item[Recombination Operator] Only the top 50\% fittest BSTs are used for recombination. Recombination is done by randomly picking two trees to combine into a single new tree. This process is described in the following paragraph.
    \item[Mutation Operator] Mutation has a 10\% chance of occurring on any tree being moved on to a subsequent generation after selection and recombination. The keys are simply shuffled and re-inserted to generate a new BST structure.
\end{description}

\subsection{Recombination}
The main novelty of this algorithm is the way in which trees do recombination. We first reduce each tree to its original list of keys. The list of keys is ordered by insertion (to the BST) order. Afterall, the tree structure is directly dependent on the order of insertion of the keys. Given the two lists of keys, we must merge the lists and remove all duplicates, so the resulting list maintains the same original set of keys. This is done based on the index of each key. Given two lists of keys $x$ and $y$, we determine a \emph{new} index for every key by randomly choosing between the key's respective index in $x$ and respective index in $y$. Specifically, if key $k_1$ is at index 3 in $x$ and at index 7 in $y$, then $k_1$ will be inserted into the ``recombined'' list at either index 3 or index 7. In cases where different keys are chosen to be inserted at the same index, they simply get inserted subsequently.


% 
\section{Results}

Figure~\ref{fig:results} shows the best, average, and worst genome fitness in the population over 300 generations for a single run of the algorithm. Since the optimal fitness value is a minimum, and larger fitness values mean a ``worse'' genome, we negate every fitness value to obtain a more intuitive graph with the minimum at the top. Results seem fairly typical for a genetic algorithm. Worst fitness values are messy and have a high variance across generations. The best fitness values converge fairly quickly and do not vary much. We assume any dips in the best fitness values are due to mutation. At the end of 300 generations, the best genome obtained a fitness of -2734. Using the classical  DP algorithm given this set of keys and frequencies, the actual optimal value is -2837, so this seems fairly close to optimal.

\begin{figure}[h]
    \centering
    \includegraphics[scale=0.8]{figures/results.pdf}
    \caption{ \small Fitness results over 300 generations. The actual optimal (calculated via DP algorithm) is plotted in blue.}
    \label{fig:results}
\end{figure}


% 
\section{Conclusion}

We have proposed an evolutionary approach for constructing near-optimal binary search trees. Like some of the previous work discussed in Section~\ref{sec-relatedwork}, we use classical evolutionary techniques to setup the problem. The novelty of the algorithm proposed in this work is the way in which crossover is implemented. Rather than attempting to combine two BSTs based directly on their structures, we consider reducing each BST to a list of keys sorted by insertion order then doing crossover. We find that the algorithm converges fairly quickly (within a few hundred generations) and is able to construct a near-optimal BST.

\subsection{Future Work}
We can propose two main directions for future work. The first is to tune the algorithm parameters in the hopes of seeing slightly better convergence. A more major direction is to reconsider the way in which crossover is implemented.



% \bibliographystyle{alpha}
% \bibliography{references}

\end{document}
