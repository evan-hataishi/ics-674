
\section{Evolutionary Algorithm}
\label{sec-algorithm}

\begin{description}
    \item[Search Space] All valid BST structures.
    \item[Representation] An $N$ node BST constructed from a set of $N$ keys and a corresponding set of $N$ frequencies.
    \item[Objective Function] Let $k$ be the set of $N$ keys, and $f$ be the set of $N$ frequencies representing a BST. The expected cost of the BST can be calculated as follows:
    \begin{equation*}
        f(k, f) = \sum_{i=1}^{N} (depth(k_i) + 1) \times f_i.
    \end{equation*}
    Therefore, we aim to minimize the expected cost, which will be some non-zero value depending on keys and frequencies.
    \item[Variation Operators] Recombination is done by merging two trees and removing duplicate nodes. Recombination is only done using parents in the top 50\% of fitness. Mutation has a 10\% chance of occurring on any tree. The keys are simply shuffled and inserted to generate a new BST structure.
    \item[Selection Operator] We use elitism to pick the top 30\% of a generation to move on automatically to the next generation. The other 70\% is formed from recombination.
\end{description}

Keys are unique and randomly generated at the start. Frequencies are picked randomly between 1 and 10 (inclusive) for each key. Recombination is difficult, for merging trees can be tricky. While merging, the resulting tree must maintain the same set of keys with no duplicates. To do this, we take the set of keys for two trees (ordered by insertion order) and randomly take the index from one of the key pairs. We then ``flatten'' the list to obtain a new list of all the keys and no duplicates.
